\documentclass[t]{beamer}
% If you don't have access to the Imperial College beamer theme, then
% commenting out the following line will produce a generic version of the
% slides. 
\usetheme{iclpt}

\usepackage{color,listings}
\usepackage{pstricks, pst-node}
\usepackage{pgfpages,xspace,array}

\newcommand{\doc}[1]{\psshadowbox[framearc=0]{#1}}
\newcommand{\program}[1]{\psframebox[framearc=.2]{#1}}

\usepackage{graphicx,stmaryrd,cancel}
\usepackage{bibentry}
\usepackage{natbib}
\usepackage{hyperref}
\usepackage{amscd}
\usepackage{booktabs}
\bibliographystyle{elsarticle-harv}

\lstset{frame=single}

\expandafter\ifx\csname natexlab\endcsname\relax\def\natexlab#1{#1}\fi
\expandafter\ifx\csname url\endcsname\relax
  \def\url#1{\texttt{#1}}\fi
\expandafter\ifx\csname urlprefix\endcsname\relax\def\urlprefix{URL }\fi

\definecolor{icdarkblue}{RGB}{23,18,134}
\definecolor{icdarkgreen}{RGB}{0,130,0}

%\setbeameroption{show notes on second screen}

\author[David A. Ham]{Dr David~A.~Ham} 
\date{24 May 2013}

\title{Verification and validation of simulation software}
                  
\institute[Imperial College London]{
  Department of Computing, Imperial College London\\
  Grantham Institute for Climate Change, Imperial College London
david.ham@imperial.ac.uk}

\begin{document}
\nobibliography{bibliography}

\begin{frame}{}
  \vfill{}

  \centering

  \Large\color{icdarkblue}\inserttitle\\
  %\normalsize\insertsubtitle\\[3ex]
  \small\color{black}\insertauthor\\[3ex]
  \footnotesize\insertinstitute

  \vfill{}

  With much material from \bibentry{Farrell2011}

\end{frame}

\begin{frame}{How do we know whether to believe the output of a model?}
  \vfill{}

  \pause
  This is fundamentally a mixed question of mathematics, software
  engineering, and application science. 
  \vfill{}

  \pause
  It's also a critical question which all computational scientists have to
  answer if they expect other scientists or society at large to take note of
  their results.
  \vfill{}
  

\end{frame}

\begin{frame}{A brief trip back to philosophy of science 101}
  
  \begin{itemize}
  \item Mathematical results are \emph{proven}. In other words, if the
    assumptions of a theorem are satisfied, then the result \emph{always}\
    follows. There is no possibility of another result.
  \item Science proceeds by \emph{hypotheses}. These can never be proven in
    the mathematical sense, but they can be \emph{falsified}\ through a
    contrary observation\footnote{\bibentry{popper1959},\\\bibentry{howden1976}}.
  \end{itemize}
  
  So what hypotheses does our software present?

\end{frame}

\begin{frame}{Verification and validation operations}

  \includegraphics[width=\textwidth]{output/verification.pdf}
  
\end{frame}

\begin{frame}{Verification and validation operations}
  
  \begin{itemize}
  \item Only the discretisation of the PDE is usually a mathematical
    operation which can be proven.
  \item Verification of software by mathematical methods is practiced in
    some fields of computing, but simulation code is typically far beyond
    their scope.
  \item The relationship between the continuous model and the physical
    system cannot be directly measured, so we must go through \emph{all}\
    the other steps. 
  \end{itemize}
  
\end{frame}

\begin{frame}{If it's not tested, it's broken.}
  
  \begin{description}
  \item[Unit tests] Tests of correct behaviour applied to the smallest
    possible unit of code.
  \item[Analytic solutions] Very strong tests of correctness of numerics and
    implementation, if you can get one!
  \item[Method of manufactured solutions] A mechanism for generating
    analytic solutions.
  \item[Regression tests] Tests against a previously computed result. Only a
    test of change, not of correctness.
  \item[Third party solution] Tests against another model. Better than
    regression tests, but only as good as the other model.
  \item[Comparisons to "real" data] The only way to validate a model, but of
    limited use in verification.
  \end{description}

\end{frame}

\begin{frame}{Software tools for verification}
  
\end{frame}

\begin{frame}{Coverage}
  
\end{frame}

\end{document}
